% Research stay plan + Gantt chart for my research stay at UBC (Winter 2026). Eléonore Perret 
% There is a lot of comments as it is my first offical latex document and I'm trying to learn with it. 

\documentclass[11pt,a4paper]{article}  % Not sure if this is the best format but let's try

% ------------------ Packages ------------------
% Not sure also about all those packages. Need to check
\usepackage[margin=2.5cm]{geometry} % Controls page layout with a margin of 2.5 cm
\usepackage{graphicx} % For inserting images  or a figure into my document. Not sure it is necessary o
\usepackage{booktabs} % apparently a package to make beautiful tables with proper spacing and rules  (Professional tables)
\usepackage{hyperref} % can create clickable links in the pdf for example for the table of content or section or URLS. 
\usepackage{enumitem} % gives me control about lists spacing and labels because latex waste a lot of vertical space by default
\usepackage{xcolor} % to use colors
\usepackage{tabularx} % to make the table go on the next line


\usepackage{pgfgantt} % Gantt chart package
\usepackage{setspace}  % COntorls the line spacing in my document. So now it is set at 1,5x normal height. 
\onehalfspacing

% ------------------ Title Info ------------------
\title{Research Stay Plan and Timeline}
\author{El\'eonore Perret}
\date{\today}

\begin{document}
\maketitle

% ------------------ Section: Overview ------------------
\section{Research Stay Overview}

\textbf{Host institution: UBC} \\
\textbf{Supervisors: Prof. Elizabeth Wolkovich and PhD Victor Van de Meersch} \\
\textbf{Goal: First draft of a joint paper on Masting at Mt. Rainier National Park} \\
\textbf{Duration:} January -- March 2026 \\

\vspace{0.5cm} % Adds vertical space (Like pressing Enter)

\noindent % Forces it to align to the left margin (which otherwise would not be the case with the default mode)
\textbf{General objective:}\\

The objectives of this research stay are to : 1) Developp a model to measure synchrony of seed production at the species and stand level at Mt. Rainier National Park and 2) Developp a manuscript or a first draft for publication. 

% ------------------ Section: Scientific Goals ------------------
\section{Scientific Goals}

\begin{enumerate}[label=\textbf{G\arabic*:}]
    \item Implement a Bayesian model to quantify interspecific synchrony in seed production.
    \item Quantify intraspecific synchrony across sites.
    \item Evaluate temporal synchrony of total seed availability from a seed predator perspective.
    \item Validate models, perform sensitivity analyses, and prepare figures for publication.
\end{enumerate}

% ------------------ Section: Output ------------------
\section{Expected Output}

\begin{itemize}
    \item Cleaned dataset
    \item Bayesian model code using Stan 
    \item Summary figures and tables
    \item Draft manuscript
\end{itemize}

% ------------------ Section: Weekly Planning ------------------
\section{Potential Weekly Planning}
% I have honestly not a clear idea of how much time each steps regarding the model take as I have never worked with those types of model. So this is tentitative and open to change :)

\begin{center}
\begin{tabularx}{\textwidth}{lX}
\toprule
\textbf{Week} & \textbf{Main Activities} \\
\midrule
Week 1--2 & Data cleaning?, getting familiar with STAN, exploratory analysis, maybe with Victor's model? \\
Week 3--4 & Model specification and prior testing + Lab retreat (potentially working already on the manuscript; Method and overall sections) \\
Week 5--6 & Model fitting and diagnostics \\
Week 7--8 & Interpretation and visualization \\
Week 9--10 & Writing and synthesis \\
Week 11--After & Wrapping up and more writing \\
\bottomrule
\end{tabularx}
\end{center}

% ------------------ Section: Gantt Chart ------------------
\newpage
\section{Timeline (Gantt Chart)}

\begin{ganttchart}[
    hgrid,
    vgrid,
    x unit=2cm, %controls widht of each month slot
    y unit title=0.6cm,
    y unit chart=0.6cm,
    bar height=0.6,
    bar/.style={fill=blue!50},
    milestone/.style={fill=red},
]{1}{3}
%Title row
\gantttitle{Jan}{1}
\gantttitle{Feb}{1}
\gantttitle{Mar}{1} \\

% Tasks
\ganttbar{Data cleaning \& exploration}{1}{1} \\
\ganttbar{Model formulation}{1}{2} \\
\ganttbar{Bayesian model fitting}{2}{2} \\
\ganttbar{Diagnostics \& validation}{2}{2} \\
\ganttbar{Visualization \& interpretation}{2}{3} \\
\ganttbar{Writing \& reporting}{3}{3} \\

% Optional milestone
\ganttbar{Final presentation / report}{3}{3}\\

\end{ganttchart}

% ------------------ Notes ------------------
\section{Notes}

Add risks, dependencies, meetings, or buffer time here.

\end{document}
